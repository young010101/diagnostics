\documentclass[UTF8]{ctexbook}

% \usepackage{ctex}
% \usepackage{fontspec}
% \setmainfont{SimSun}

\usepackage{geometry}
\geometry{left=2.5cm,top=2.5cm}

% \usepackage[draft]{graphicx}
\usepackage{graphicx}
% why it didn't useful?
\graphicspath{./figure/}

\setcounter{secnumdepth}{5}

\renewcommand{\thesection}{第\zhnum{section}节}
\renewcommand{\thesubsection}{\zhnum{subsection}}
\renewcommand{\thesubsubsection}{(\zhnum{subsubsection})}
\renewcommand{\theparagraph}{\arabic{paragraph}.}
\renewcommand{\thetable}{\arabic{part}-\arabic{chapter}-\arabic{table}}
\renewcommand{\thefigure}{\arabic{part}-\arabic{chapter}-\arabic{figure}}

\usepackage{array}
\newcolumntype{L}[1]{>{\vspace{0.5em}\begin{minipage}{#1}\raggedright\let\newline\\
\arraybackslash\hspace{0pt}}m{#1}<{\end{minipage}\vspace{0.5em}}}
\newcolumntype{R}[1]{>{\vspace{0.5em}\begin{minipage}{#1}\raggedleft\let\newline\\
\arraybackslash\hspace{0pt}}m{#1}<{\end{minipage}\vspace{0.5em}}}
\newcolumntype{C}[1]{>{\vspace{0.5em}\begin{minipage}{#1}\centering\let\newline\\
\arraybackslash\hspace{0pt}}m{#1}<{\end{minipage}\vspace{0.5em}}}

\newcommand{\kouluo}{\hbox{\scalebox{0.45}[1]{口}\kern-.1em\scalebox{0.6}[1]{罗}}}

\usepackage{multirow}

\usepackage{enumerate}

\usepackage{siunitx}
\usepackage{mhchem}

% use super table
% \usepackage{array,tabularx}
\usepackage{tabularx}

\begin{document}
\part{常见症状}
\part{问诊}
\part{体格检查}

\chapter{基本方法}
\section{视诊}
\section{触诊}
\section{叩诊}
\section{听诊}
\section{嗅诊}

\chapter{一般检查}
\begin{enumerate}[(1)]
    \item 腋测法: 5分钟后读数。正常值为36$\sim$37~$^{\circ}$C
    \item 口测法: 5分钟后读数。正常值为36.3$\sim$37.2~$^{\circ}$C
    \item 肛测法: 5分钟后读数。正常值为36.5$\sim$37.7~$^{\circ}$C。
\end{enumerate}

\section{皮肤}
\begin{figure}[htbp]
    \centering
    \includegraphics{figure/颈部淋巴结群.jpg}
\end{figure}
\begin{figure}[htbp]
    \centering
    \includegraphics{figure/表浅淋巴结-上肢.jpg}
\end{figure}
\chapter{头部检查}
\section{头发和头皮}
\section{头颅}
头颅的大小:

用头围衡量。

用软尺自眉间绕到颅后通过枕骨隆突。

% Table generated by Excel2LaTeX from sheet 'Sheet1'
\begin{table}[htbp]
    \centering
    \caption{临床常见头颅的大小异常或畸形}  \label{t}
    \begin{tabular}{ll}
        \hline
        小颅   & 囟门过早闭合:常伴有大脑发育不全、智力障碍                           \\
        尖颅   & Apert综合征: 先天性尖颅并指(趾)畸形,                              \\
        方颅   & 小儿佝偻病或先天梅毒                                                 \\
        长颅   & Manfan综合征及肢端肥大症                                             \\
        巨颅   & 脑积水:顶、颞、枕部突出膨大呈圆形;颅内压增高压迫眼球出现落日现象。 \\
        变形颅 & 畸形性骨炎:以颅骨增大变形为特征,伴有长骨的骨质增厚与弯曲。         \\
        \hline
    \end{tabular}%
    
\end{table}%


\section{颜面及其器官}
\subsection{眼}
\subsubsection{眼的功能检查}

\paragraph{视力(visual acuity)}
\paragraph{视野(visual fields)} 是当眼球向正前方固视不动时所见的空间范围,与中央视力相对
而言,它是周围视力,是检查黄斑中心凹以外的视网膜功能。

\paragraph{色觉(color sensation)}
\paragraph{立体视的检查}
\subsubsection{外眼检查}
\paragraph{眼脸(eyelids)}
\begin{enumerate}[(1)]
    \item 脸内翻(entropion) : 由于瘢痕形成使脸缘向内翻转,见千沙眼。
    \item 上睑下垂(ptosis) :

          双侧上睑下垂见于先天性上脸下垂、重症肌无力;

          单侧上睑下垂见于蛛网膜下腔出血、白喉、脑脓肿、脑炎、外伤等引起的动眼神经麻痹。
    \item 眼脸闭合障碍:

          双侧眼脸闭合障碍可见千甲状腺功能亢进症;

          单侧闭合障碍见于面神经麻痹。
    \item 眼脸水肿: 眼脸皮下组织疏松,轻度或初发水肿常在眼脸表现出来。常见原因为肾炎、慢性肝病、营养不良、贫血血管神经性水肿等。此外,还应注意眼脸有无包块、压痛、倒睫等。
\end{enumerate}
\subsubsection{眼前节检查}
\paragraph{瞳孔(pupil)}瞳孔是虹膜中央的孔洞,正常直径为3$\sim$4~mm 。

瞳孔缩小( 瞳孔括约肌收缩),是由动眼神经的副交感神经纤维支配;

瞳孔扩大(瞳孔扩大肌收缩),是由交感神经支配。
\begin{enumerate}[(1)]
    \item 瞳孔的形状与大小:正常为圆形,双侧等大。

          病理情况下,瞳孔缩小见于虹膜炎症、中毒(有机磷类农药)、药物反应(毛果芸香碱、吗啡、氯丙唉)等。
\end{enumerate}
\subsubsection{内眼}

\chapter{颈部检查}

颈静脉  颈静脉充盈的高度反映静脉压水平。一般多取右侧颈静脉进行观察。

正常人立位或坐位时颈外静脉(简称颈静脉)常不显露。

平卧时可见颈外静脉充盈,充盈水平限于锁骨上缘至下颌角距离的下2/3内。

取30~45度半卧位,颈静脉充盈程度超过正常水平,为颈静脉怒张,提示静脉压增高。

静脉压异常增高:见于右心衰竭、缩窄性心包炎、心包积液或上腔静脉阻塞综合征。


颈动脉搏动:正常人在安静状态下不易看到颈动脉搏动,只在剧烈活动后心搏出量增加时可见。如在安静状态下出现颈动脉的明显搏动,则多见主动脉关闭不全、甲状腺功能亢进及严重贫血病人。

颈静脉搏动:在正常情况下不会出现颈静脉搏动,在三尖瓣关闭不全伴有颈静脉怒张时可看到。

动脉和静脉搏动的鉴别:一般静脉搏动柔和,范围弥散,触诊时无搏动感;动脉搏动较强劲,为膨胀性,触诊时搏动感明显。

颈部血管听诊

颈部大血管处:收缩期杂音应考虑颈动脉狭窄。

锁骨上窝处:听到杂音,可能为锁骨下动脉狭窄。

右锁骨上窝处听到连续性“营营”样杂音:则可能为颈静脉流入上腔静脉口径较宽的球部所产生,这种杂音是生理性的,用手指压迫颈静脉后可消失。

\subsection{甲状腺}

甲状腺肿大可分为三度:

Ⅰ度:不能看出肿大但能触及者;

Ⅱ度:能看到肿大又能触及,但在胸锁乳突肌外缘以内者;

Ⅲ度:超过胸锁乳突肌外缘者。

听诊

当甲状腺肿大时,用\underline{钟型}听诊器直接放在肿大的甲状腺上,如能听到低调的连续性血管杂音,对诊断甲状腺功能亢进很有帮助。


\subsection{甲状腺}
\subsection{甲状腺}
\subsection{甲状腺}

\chapter{胸部检查}(P142)
\begin{itemize}
    \item 掌握胸部体表标志
    \item 掌握肺部视、触、叩、听诊的正常和异常现象、其发生机制和临床意义
    \item 熟悉胸壁、胸廓和乳房的检查。呼吸系统常见病的主要症状及体征(大叶性肺炎、支气管哮喘、胸腔积液等)。
    \item 运用肺部查体帮助诊断呼吸系统常见疾病
\end{itemize}

\section{胸部的体表标志}
\subsection{骨骼标志}
\paragraph{胸骨柄}
\paragraph{胸骨上切迹}
\paragraph{胸骨角}
\paragraph{腹上角}
\paragraph{剑突}
\paragraph{肋骨}
\paragraph{肋间隙}
\paragraph{肩胛骨}
\paragraph{脊柱棘突}
\paragraph{肋脊角}

\subsection{垂直线标志}
\subsection{自然陷窝和解剖区域}
\subsection{肺和胸膜的界限}
\section{胸壁、胸廓与乳房}
\subsection{乳房}
\subsubsection{视诊}
\subsubsection{乳房的常见病变}
\paragraph{急性乳腺炎}乳房红、肿、热、痛, 常局限于一侧乳房的某一象限。触诊有硬结包块, 伴寒战、发热及出汗等全身中毒症状, 常发生于哺乳期妇女, 但亦见于青年女性和男子。
\paragraph{乳腺肿瘤}应区别良性或恶性, 乳腺癌一般无炎症表现, 多为单发并与皮下组织粘连, 局部皮肤呈橘皮样, 乳头常回缩。多见于中年以上的妇女, 晚期每伴有腋窝淋巴结转移。良性肿瘤则质较柔韧或中硬, 界限清楚并有一定活动度, 常见者有乳腺纤维瘤等。
\section{肺和胸膜}
\subsection{视诊}
\subsubsection{呼吸运动}
高碳酸血症可直接抑制呼吸中枢使呼吸变浅。

一般成人静息呼吸时, 潮气量约为500ml 。

正常男性和儿童的呼吸以膈肌运动为主, 胸廓下部及上腹部的动度较大, 而形成腹式呼吸;女性的呼吸则以肋间肌的运动为主, 故形成胸式呼吸。

''三凹征''(three depressions sign)。

\subsubsection{呼吸频率}
正常成入静息状态下, 呼吸为12$\sim$20次/分,

\subsubsection{呼吸节律}


\begin{table}[!htbp]
    \centering
    \caption{常见异常呼吸类型的病因和特点}\label{table001}

    % \usepackage{array,tabularx}
    \begin{tabular}{l*{2}{L{.34\textwidth}}}
        \hline
        类型         & 特点                                                                       & 病因                                                           \\
        \hline
        呼吸停止     & 呼吸消失                                                                   & 心脏停搏                                                       \\
        比奥呼吸     & 规则呼吸后出现长周期呼吸停止又开始呼吸                                     & 颅内压增高, 药物引起呼吸抑制, 大脑损害(通常于延髓水平)         \\
        陈施呼吸     & 不规则呼吸呈周期性, 呼吸频率和深度逐渐增加和逐渐减少导致呼吸暂停相交替出现 & 药物引起的呼吸抑制, 充血性心力衰竭, 大脑损伤(通常于脑皮质水平) \\
        库斯莫尔呼吸 & 呼吸深慢                                                                   & 代谢性酸中毒                                                   \\
        \hline
    \end{tabular}

\end{table}
\begin{table}[htbp]
    \centering
    \begin{tabularx}{0.99\textwidth}%
        {l*{2}{>{\raggedright\arraybackslash}X}}    % why is raggedright instead of left?
        % command \arraybackslash is use to damage of change //

        \hline
        类型         & 特点                                                                       & 病因                                                           \\
        \hline
        呼吸停止     & 呼吸消失                                                                   & 心脏停搏                                                       \\
        比奥呼吸     & 规则呼吸后出现长周期呼吸停止又开始呼吸                                     & 颅内压增高, 药物引起呼吸抑制, 大脑损害(通常于延髓水平)         \\
        陈施呼吸     & 不规则呼吸呈周期性, 呼吸频率和深度逐渐增加和逐渐减少导致呼吸暂停相交替出现 & 药物引起的呼吸抑制, 充血性心力衰竭, 大脑损伤(通常于脑皮质水平) \\
        库斯莫尔呼吸 & 呼吸深慢                                                                   & 代谢性酸中毒                                                   \\
        \hline
    \end{tabularx}
\end{table}


\subsection{触诊}
\subsubsection{语音震颤}

语音震颤减弱或消失
\begin{enumerate}
    \item 肺泡内含气量过多, 如慢性阻塞性肺疾病;
    \item 支气管阻塞, 如阻塞性肺不张;
    \item 大量胸腔积液或气胸;
    \item 胸膜显著增厚粘连;
    \item 胸壁皮下气肿。
\end{enumerate}

语音震颤增强
\begin{enumerate}
    \item 大叶性肺炎实变期、大片肺梗死等;
    \item 空洞型肺结核、肺脓肿等。
\end{enumerate}

\subsubsection{胸膜摩擦感}
胸膜摩擦感(pleural friction fremitus)指当急性胸膜炎时, 因纤维蛋白沉着于两层胸膜, 使其表面变得粗糙, 呼吸时脏层胸膜和壁层胸膜相互摩擦, 可由检查者的手感觉到, 故称为胸膜摩擦感。

\subsection{叩诊}
\subsubsection{正常叩诊音}
\paragraph{正常胸部叩诊音}正常胸部叩诊为清音, 其音响强弱和高低与肺脏含气量的多寡、胸壁的厚薄以及邻近器官的影响有关。
\paragraph{侧卧位的胸部叩诊}
侧卧位时由于一侧胸部靠近床面对叩诊音施加影响, 故近床面的胸部可叩得一条相对浊音或实音带。在该带的上方区域由于腹腔脏器的压力影响, 使靠近床面一侧的膈肌升高, 可叩出一粗略的浊音三角区。

因侧卧时脊柱弯曲, 使靠近床面一侧的胸廓肋间隙增宽, 而朝上一侧的胸廓肋骨靠拢肋间隙变窄。故于朝上的一侧的肩胛角尖端处可叩得一相对的浊音区, 撤去枕头后由于脊柱伸直, 此浊音区即行消失。

\begin{figure}[htbp]
    \centering
    \includegraphics[width=0.5\textwidth]{figure/侧卧胸部叩诊.png}
    \caption{侧卧位的叩诊音}
\end{figure}

\subsubsection{胸部异常叩诊音}
正常肺脏的清音区范围内, 如出现浊音、实音、过清音或鼓音时则为异常叩诊音(P158)。

肺部大面积含气量减少的病变, 如肺炎、肺不张、肺结核、肺梗死、肺水肿及肺硬化等;和肺内不含气的占位病变, 如肺肿瘤、肺棘球蚴病或囊虫病、未液化的肺脓肿等;以及胸腔积液, 胸膜增厚等病变, 叩诊均为浊音或实音。

肺张力减弱而含气量增多时, 如慢性阻塞性肺疾病等, 叩诊呈过清音(hyperresonance) 。肺内空腔性病变如其腔径大于3$\sim$4 cm, 且靠近胸壁时, 如空洞型肺结核、液化了的肺脓肿和肺囊肿等, 叩诊可呈鼓音。
\subsection{听诊}

\subsubsection{正常呼吸音}
\paragraph{支气管肺泡呼吸音}正常人于胸骨两侧第1、2肋间隙, 肩胛间区第3、4胸椎水平以及肺尖前后部可听及支气管肺泡呼吸音。
% Table generated by Excel2LaTeX from sheet 'Sheet1'
\begin{table}[htbp]
    \centering
    \caption{4种正常呼吸音特征的比较}
    \begin{tabular}{l*{4}{L{0.18\textwidth}}}
        \hline
        特征         & 气管呼吸音 & 支气管呼吸音 & 支气管肺泡呼吸音 & 肺泡呼吸音   \\
        \hline
        强度         & 极响亮     & 响亮         & 中等             & 柔和         \\
        音调         & 极高       & 高           & 中等             & 低           \\
        吸:呼        & 1:1        & 1:3          & 1:1              & 3:1          \\
        性质         & 粗糙       & 管样         & 沙沙声, 但管样   & 轻柔的沙沙声 \\
        正常听诊区域 & 胸外气管   & 胸骨柄       & 主支气管         & 大部分肺野   \\
        \hline
    \end{tabular}%
    \label{tab:addlabel}%
\end{table}%

\subsubsection{\kouluo 音}

呼吸音以外的附加音, 该音正常情况下并不存在, 故非呼吸音的改变。

\paragraph{湿\kouluo 音}系由于吸气时气体通过呼吸道内的分泌物如渗出液、痰液、血液、黏液和脓液等, 形成的水泡破裂所产生的声音, 又称水泡音。或认为由于小支气管壁因分泌物黏着而陷闭, 当吸气时突然张开重新充气所产生的爆裂音。

\begin{enumerate}[(1)]
    \item 特点:为呼吸音外的附加音, 断续而短暂, 部位较恒定, 性质不易变。
    \item ff
\end{enumerate}

\subsubsection{语音共振}
语音共振的产生方式与语音震颤基本相同。(触诊、听诊)

\section{呼吸系统常见疾病的主要症状和体征}
\subsection{大叶性肺炎}
大叶性肺炎是大叶性分布的肺脏炎性病变。病理改变可分为三期, 即充血期、实变期及消散期。

\paragraph{症状}起病多急骤, 表现为寒战高热, 头痛, 全身肌肉酸痛, 患侧胸痛, 呼吸增快, 咳嗽, 咳铁锈色痰。

\paragraph{体征}
\subsection{气胸}

\begin{table}[htbp]
    \centering
    \caption{肺与胸膜常见疾病的体征}
    \begin{tabular}{*{9}{L{0.075\textwidth}}}
        \hline
        \multirow{2}{*}{疾病} & \multicolumn{2}{c}{视诊} & \multicolumn{2}{c}{触诊} & 叩诊     & \multicolumn{3}{c}{听诊}                                                         \\
                              & 胸廓                     & 呼吸动度                 & 气管位置 & 语音震颤                 & 音响       & 呼吸音       & \kouluo 音   & 语音共振   \\
        \hline
        大叶性肺炎            & 对称                     & 患侧减弱                 & 正中     & 患侧增强                 & 浊音       & 支气管呼吸音 & 湿\kouluo 音 & 患侧增强   \\
        慢性阻塞性肺          & 桶状                     & 双侧减弱                 & 正中     & 双侧减弱                 & 过清音     & 减弱         & 多无         & 减弱       \\
        哮喘                  & 对称                     & 双侧减弱                 & 正中     & 双侧减弱                 & 过消音     & 减弱         & 干\kouluo 音 & 减弱       \\
        肺水肿                & 对称                     & 双侧减弱                 & 正中     & 正常或减弱               & 正常或浊音 & 减弱         & 湿\kouluo 音 & 正常或减弱 \\
        肺不张                & 患侧平坦                 & 患侧减弱                 & 移向患侧 & 减弱或消失               & 浊音       & 减弱或消失   & 无           & 减弱或消失 \\
        胸腔积液              & 患侧饱满                 & 患侧减弱                 & 移向健侧 & 减弱或消失               & 实音       & 减弱或消失   & 无           & 减弱       \\
        气胸                  & 患侧饱满                 & 患侧减弱或悄失           & 移向健侧 & 减弱或消失               & 鼓音       & 减弱或消失   & 无           & 减弱或消失 \\

        \hline
    \end{tabular}
\end{table}

\section{心脏检查}
\subsection{视诊}
\subsubsection{胸廓畸形}

\subsubsection{心尖搏动}
心尖搏动(apical impulse)主要由千心室收缩时心脏摆动,心尖向前冲击前胸壁相应部位而形成。正常成人心尖搏动位于第5肋间, 左锁骨中线内侧0.5$\sim$1.0 cm,搏动范围以直径计算为2.0$\sim$2.5 cm。
\subsection{触诊}
\subsection{叩诊}(P147)
\begin{table}[htbp]
    \caption{心浊音界改变的心脏因素和临床常见疾病}
    \begin{tabular}{lll}
        \hline
        因素           & 心浊音界                                             & 临床常见疾病         \\
        \hline
        左心室增大     & 向左下增大,心腰加深,心界似靴形(图\ref{zhudongmai}) & 主动脉瓣关闭不全等   \\
        右心室增大     & 轻度增大:绝对浊音界增大,相对浊音界无明显改变        & 肺源性心脏病或房间隔 \\
                       & 显著增大:心界向左右两侧增大                          & 缺损等               \\
        左、右心室增大 & 心浊音界向两侧增大,且左界向左下增大,称普大型       & 扩张型心肌病等       \\
        左心房增大或合 & 左房显著增大:胸骨左缘第3 肋间心界增大, 心腰消失     & 二尖瓣狭窄等         \\
        并肺动脉段扩大 & 左房与肺动脉段均增大: 胸骨左缘第2 、3 肋间心界增大, &                      \\
                       & 心腰更为丰满或膨出, 心界如梨形(图\ref{erjianban})   &                      \\
        主动脉扩张     & 胸骨右缘第l 、2 肋间浊音界增宽,常伴收缩期搏动       & 升主动脉瘤等         \\
        心包积液       & 两侧增大,相对、绝对浊音界几乎相同,并随体位而改变, & 心包积液             \\
                       & 坐位时心界呈三角形烧瓶样, 卧位时心底部浊音增宽      &                      \\
        \hline
    \end{tabular}
\end{table}

\begin{figure}[htbp]
    \centering
    \includegraphics{figure/主动脉瓣关闭不全的心浊音界.jpg}
    \caption{主动脉瓣关闭不全的心浊音界(靴形心)}\label{zhudongmai}
\end{figure}

\begin{figure}[htbp]
    \centering
    \includegraphics{figure/二尖瓣狭窄的心浊音界.jpg}
    \caption{二尖瓣狭窄的心浊音界(梨形心)}\label{erjianban}
\end{figure}

\subsection{听诊}
\subsubsection{听诊内容}
\subsubsection{听诊内容}
\subsubsection{听诊内容}
\paragraph{心音(heart sound)}
\paragraph{心音(heart sound)}
\paragraph{心音(heart sound)}按其在心动周期中出现的先后次序,可依次命名为第一心音(first heart sound, S$_1$)、第
\paragraph{心音的改变及其临床意义}
\begin{enumerate}[(1)]
    \item 心音强度改变: 除肺含气量多少、胸壁或胸腔病变等心外因素以及是否有心包积液外, 影响心音强度的主要因素是心肌收缩力与心室充盈程度(影响心室内压增加的速率) ,以及瓣膜位置的高低、瓣膜的结构和活动性等。

          \begin{enumerate}
              \item[1)] 第一心音强度的改变:主要决定因素是心室内压增加的速率,心室内压增加的速率越快, S$_1$越强; 其次受心室开始收缩时二尖瓣和三尖瓣的位置和上述其他因素影响。

                  S$_1$增强: 常见于二尖瓣狭窄。由千心室充盈减慢减少, 以致在心室开始收缩时二尖瓣位置低垂
              \item[2)] 第二心音强度的改变:体或肺循环阻力的大小和半月瓣的病理改变是影响S$_2$的主要因素。S$_2$有两个主要部分即主动脉瓣部分(A$_2$) 和肺动脉瓣部分(P$_2$)
          \end{enumerate}
    \item f

\end{enumerate}

\chapter{腹部检查}
\chapter{肛门与直肠检查}
\chapter{脊柱}

熟悉脊柱、四肢的检查方法

熟悉异常体征的临床意义


思考题
\begin{enumerate}
    \item 脊柱检查的内容有哪些?简述其检查方法及各种病变的临床意义。
    \item 四肢及关节的形态改变有哪些?各有何临床意义
    \item 简述四肢运动功能检查的内容、方法及运动功能障碍常见的病因
\end{enumerate}

\section{脊柱检查}

检查内容:弯曲度、有无畸形、活动度是否受限、有无压痛及叩击痛。


常用检查方法
(视、触、叩)

背面视诊

侧面视诊

脊柱压痛与叩击痛

\section{四肢与关节}

检查方法

视诊

触诊

特殊情况下采用叩诊和听诊


\chapter{神经}

\part{实验诊断}
\part{辅助检查}
\chapter{心电图}(P1...)

掌握
\begin{itemize}
    \item 心脏的特殊传导系统及心电图各波段的组成;
    \item 常规心电图导联及电极的放置部位;
    \item 心电图参数的测量方法;
    \item 正常心电图波形的特点和正常值;
    \item 临床上常见的异常心电图表现(房室肥大,心肌缺血与心肌梗死,期前收缩,室上性心动过速,室性心动过速,扭转型室速,心房扑动与心房颤动,房室传导阻滞,预激综合征,高血钾与低血钾)。
\end{itemize}

熟悉
\begin{itemize}
    \item 心电图的分析步骤及临床应用;
    \item 心律失常分类;
    \item 窦性心律失常的几种表现;
    \item 窦房阻滞的心电图表现;
    \item 左、右束支阻滞及其分支阻滞的心电图表现;
    \item 逸搏与逸搏心律。
\end{itemize}

\section{临床心电学的基本知识}
\subsection{心电图产生原理}
\subsection{心电图各波段的组成和命名}

% Table generated by Excel2LaTeX from sheet 'Sheet1'
\begin{table}[htbp]
    \centering
    \caption{Add caption}
    \begin{tabular}{ll}
        \hline
        P波       & 心房的除极化                   \\
        PR段      & 从心房除极至心室除极所用的时间 \\
        QRS 波群  & 心室的除极化                   \\
        ST段和T波 & 心室的复极化                   \\
        QT间期    & 整个从心室除极到复极所用的时间 \\
        \hline
    \end{tabular}%
    \label{tabel}%
\end{table}%

\subsection{心电图导联体系}
\paragraph{肢体导联(limb leads)}包括标准肢体导联I 、II 、III及加压肢体导联aVR 、aVL 、aVF 。肢体导联的电极主要放置千右臂(R)、左臂(L)、左腿(F), 连接此三点即成为所谓Einthoven 三角(图5-l - 8A 、B) 。
\paragraph{胸导联(chest leads)}

v1 位于胸骨右缘第4 肋间;

v2 位于胸骨左缘第4 肋间

V3位于V点V4 两点连线的中点

V4 位于左锁骨中线与第5 肋间相交处;

Vs 位于左腋前线与v4同一水平处;

v6 位于左腋中线与v4 同一水平处。
\section{心电图的测量和正常数据}
\subsection{心电图测量}

当走纸速度为25mm/s 时, 每两条纵线间(1mm) 表示0.04 秒( 即40 毫秒),

当标准电压\SI{1}{mV}= 10mm 时, 两条横线间(1 mm) 表示0.1 mV 。

\subsubsection{心率的测量}
在安静清醒的状态下,正常心率范围在60 -100 次/分。

R波  首先出现的位于参考水平线以上的正向波

Q波  R波之前的负向波

S波   R波之后的第一个负向波

R’波  S波之后的正向波

S’波   R’波之后的负向波

QS波    QRS波只有负向波

若振幅>0.5mv,用大写字母表示

若振幅<0.5mv,用小写字母表示

振幅小可称为q、r、s、r’、s’

\subsection{正常心电图波形特点和正常值}
\paragraph{P波}代表心房肌除极的电位变化。
\begin{enumerate}[(1)]
    \item 时间:正常人P 波时间一般小于0.12秒。
    \item 振幅: P 波振幅在肢体导联一般小千0. 25mV, 胸导联一般小于0.2mV 。
\end{enumerate}
\paragraph{QRS 波群}代表心室肌除极的电位变化。
\begin{enumerate}[(1)]
    \item 不超过0. 11 秒,
    \item 胸导联,V5 、V6 导联不超过2.5mV 。肢体导联,I 导联的R 波小于l. 5mV。

          6 个肢体导不应都小于0.5mV,6 个胸导不应都小于0. 8mV,
    \item R 峰时间(R peak time): R峰时间延长见于心室肥大,预激综合征及心室内传导阻滞。
    \item Q 波:不超过0.03秒(除III和aVR 导联外) 。Q 波深度不超过同导联R 波振幅的1/4 。
\end{enumerate}
\paragraph{ST段}不超过0. 05mV 。
\paragraph{T波}代表心室快速复极时的电位变化。

\section{心房肥大和心室肥厚}
\subsection{心室肥厚}
\subsubsection{左心室肥厚}
面向左心室的导联(I 、aVL 、Vs 和V6) 其R 波振幅增加

而面向右心室的导联($V_1$和V2) 则出现较深的S 波

\section{心肌缺血与ST-T 改变}

\section{心肌梗死}

\section{心律失常}
\subsection{概述}
\subsection{窦性心律及窦性心律失常}
\subsection{期前收缩}
\subsection{逸搏与逸搏心律}
\subsection{异位性心动过速}
\subsection{扑动与颤动}

\subsection{传导异常}
\subsubsection{预激综合征}

\section{电解质紊乱和药物影响}
\section{心电图的分析方法和临床应用}
\chapter{其他常用心电学检查}

\part{病例书写}
hello

\listoffigures
\listoftables
\end{document}

